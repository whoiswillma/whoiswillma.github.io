%%%%%%%%%%%%%%%%%%%%%%%%%%%%%%%%%%%%%%%
% Deedy - One Page Two Column Resume
% LaTeX Template
% Version 1.1 (30/4/2014)
%
% Original author:
% Debarghya Das (http://debarghyadas.com)
%
% Original repository:
% https://github.com/deedydas/Deedy-Resume
%
% IMPORTANT: THIS TEMPLATE NEEDS TO BE COMPILED WITH XeLaTeX
%
% This template uses several fonts not included with Windows/Linux by
% default. If you get compilation errors saying a font is missing, find the line
% on which the font is used and either change it to a font included with your
% operating system or comment the line out to use the default font.
% 
%%%%%%%%%%%%%%%%%%%%%%%%%%%%%%%%%%%%%%
% 
% TODO:
% 1. Integrate biber/bibtex for article citation under publications.
% 2. Figure out a smoother way for the document to flow onto the next page.
% 3. Add styling information for a "Projects/Hacks" section.
% 4. Add location/address information
% 5. Merge OpenFont and MacFonts as a single sty with options.
% 
%%%%%%%%%%%%%%%%%%%%%%%%%%%%%%%%%%%%%%
%
% CHANGELOG:
% v1.1:
% 1. Fixed several compilation bugs with \renewcommand
% 2. Got Open-source fonts (Windows/Linux support)
% 3. Added Last Updated
% 4. Move Title styling into .sty
% 5. Commented .sty file.
%!TEX encoding = UTF-8 Unicode
%%%%%%%%%%%%%%%%%%%%%%%%%%%%%%%%%%%%%%%
%
% Known Issues:
% 1. Overflows onto second page if any column's contents are more than the
% vertical limit
% 2. Hacky space on the first bullet point on the second column.
%
%%%%%%%%%%%%%%%%%%%%%%%%%%%%%%%%%%%%%%

\documentclass[]{deedy-resume-openfont}

\newcommand{\myurl}[1]{
	\urlstyle{same}\url{#1}
}

\newcommand{\custombold}[1]{
	#1
}

\colorlet{SECTIONTITLECOLOR}{RoyalBlue}
\let\sectionold\section
\renewcommand{\section}[1]{
	\sectionold{\textcolor{sectiontitlecolor}{#1}}
}

\begin{document}

%%%%%%%%%%%%%%%%%%%%%%%%%%%%%%%%%%%%%%
%
%     LAST UPDATED DATE
%
%%%%%%%%%%%%%%%%%%%%%%%%%%%%%%%%%%%%%%
% \lastupdated

%%%%%%%%%%%%%%%%%%%%%%%%%%%%%%%%%%%%%%
%
%     TITLE NAME
%
%%%%%%%%%%%%%%%%%%%%%%%%%%%%%%%%%%%%%%


\namesection{}{\textcolor{RoyalBlue}{William Ma}}{ \urlstyle{same}\url{http://whoiswillma.github.io} \\
wm274@cornell.edu | (845) 337-7521
}

%%%%%%%%%%%%%%%%%%%%%%%%%%%%%%%%%%%%%%
%
%     COLUMN ONE
%
%%%%%%%%%%%%%%%%%%%%%%%%%%%%%%%%%%%%%%

\begin{minipage}[t]{0.33\textwidth} 

%%%%%%%%%%%%%%%%%%%%%%%%%%%%%%%%%%%%%%
%     SUMMARY
%%%%%%%%%%%%%%%%%%%%%%%%%%%%%%%%%%%%%%

\section{Introduction}

\location{ I am an open-source contributor to a package manager for macOS, an experienced iOS developer working on an on-campus dining app, and an academic student interested in compilers, computability theory, and algorithms. }

% \location{ I am an open-source contributor interested in developer tools and solid testing. I contributed to Homebrew by adding license information to nearly half of core Homebrew packages, and unified the commands of Homebrew and its sibling project, \texttt{brew cask} }

% \location{ I am an experienced iOS developer and design enthusiast. I am a project manager and frontend developer for Eatery, an app used by 50\% of students at Cornell. }

% \location{ I am interested in compilers, computability theory, and algorithmic theory. The Xi Compiler I helped build compiles a type-checked LALR language into x86 assembly with dataflow analysis-based optimizations. }

%%%%%%%%%%%%%%%%%%%%%%%%%%%%%%%%%%%%%%
%     EDUCATION
%%%%%%%%%%%%%%%%%%%%%%%%%%%%%%%%%%%%%%

\section{\textbf{Education}}

\subsection{Cornell University}

\descript{BS, Computer Science}
\location{Expected May 2022 | Ithaca, NY}
College of Engineering \\
Dean's List (3 semesters) \\
\location{ GPA: 4.3 }
\sectionsep

\subsection{Arlington High School}
\location{2014 - 2018 | Lagrangeville, NY}

%%%%%%%%%%%%%%%%%%%%%%%%%%%%%%%%%%%%%%
%     LINKS
%%%%%%%%%%%%%%%%%%%%%%%%%%%%%%%%%%%%%%

%%%%%%%%%%%%%%%%%%%%%%%%%%%%%%%%%%%%%%
%     COURSEWORK
%%%%%%%%%%%%%%%%%%%%%%%%%%%%%%%%%%%%%%

\section{Coursework}
\subsection{Computer Science}
Analysis of Algorithms, Compilers, Functional Programming, Object-Oriented Design and Data Structures, Theory of Computing, Discrete Structures
\sectionsep

\subsection{Business}
Financial and Managerial Accounting, Macroeconomics
\sectionsep

\subsection{Mathematics}
Linear Algebra, Multivariable Calculus

%%%%%%%%%%%%%%%%%%%%%%%%%%%%%%%%%%%%%%
%     SKILLS
%%%%%%%%%%%%%%%%%%%%%%%%%%%%%%%%%%%%%%

\section{Technical Skills}
\subsection{Proficient}
Swift, Java, OCaml, Xcode, UIKit, Animations with UIKit, Core Data, Core Location, MapKit, \LaTeX{}, Git, Vim, Ubuntu
\sectionsep
\subsection{Experienced}
Objective C, Python, Verilog, HTML, CSS, Android Studio, Web Development, Nginx, VMware ESXi

\section{Activites}
% \vspace{\topsep} % Hacky fix for awkward extra vertical space
Cornell University Glee Club


%%%%%%%%%%%%%%%%%%%%%%%%%%%%%%%%%%%%%%
%
%     COLUMN TWO
%
%%%%%%%%%%%%%%%%%%%%%%%%%%%%%%%%%%%%%%

\end{minipage} 
\hfill
\begin{minipage}[t]{0.66\textwidth} 

%%%%%%%%%%%%%%%%%%%%%%%%%%%%%%%%%%%%%%
%     EXPERIENCE
%%%%%%%%%%%%%%%%%%%%%%%%%%%%%%%%%%%%%%

% \vspace{\topsep} % Hacky fix for awkward extra vertical space

\section{Homebrew} \runsubsection{Open-source contributor} \\
\descript{The Missing Package Manager for macOS (or Linux)}
\descript{Summer 2020 \textemdash \ Present}
\vspace{\topsep} % Hacky fix for awkward extra vertical space 
\begin{tightemize}
\item \textbf{Unified 8 brew and brew cask commands; Added cask support to 3 brew commands}
\item \textbf{Added license information to 50\% (ct. $\approx$ 2500) of homebrew-core packages}
\item Part of the inaugural MLH Fellowship
\item \textcolor{darkgray}{\textit{Github:} \myurl{https://github.com/homebrew}}
\item \textcolor{darkgray}{\textit{Technologies:} Ruby, Rspec, DSLs}
\end{tightemize}

\section{Eatery}
\runsubsection{iOS Developer \& Project Manager} \\
\descript{Find food at cornell}
\descript{Fall 2018 \textemdash \ Present}
\begin{tightemize}
\item \textbf{Grew user base by 20\% during 2019}
\item \textbf{7,200 Monthly Active Users, 2,500 Daily Active Users} 
\item Created interactive bar-chart view for Popular Times feature
\item Architected and implemented a large feature: Collegetown Eateries
\item Implemented design changes; refactored legacy UI code
\item Collaborated on a team with iOS, Android, backend developers, as well as designers and marketers
\item \textcolor{darkgray}{\textit{Github:} \myurl{https://github.com/cuappdev/eatery}}  \\
\item \textcolor{darkgray}{\textit{Technologies:} iOS, Swift 4.2, UIKit, Apollo} 
\end{tightemize}

\section{Xi Compiler}
\descript{Compiler for imperative language into x86 assembly}
\descript{Spring 2020 for CS 4120}
\begin{tightemize}
\item \textbf{Implemented dataflow analysis optimizations: Register allocation, copy propagation, dead code elimination, loop unrolling}
\item \textbf{Developed test suite with unit tests, performance tests, and integration tests.} Used code coverage tools throughout development.
\item Built lexing with JFlex, parsing with CUP, type-checking, syntax-directed IR generation, dataflow-based optimization, ASM generation with tiling
\item Used visitor pattern for AST to MIR to LIR translation
\item \textcolor{darkgray}{\textit{Technologies:} Java, JUnit testing, Gradle, \texttt{gdb}, JFlex, CUP}
\end{tightemize}

\section{Cello Mute}
\runsubsection{Independent iOS App} \\
\descript{The tools musicians need, all in one}
\descript{Summer 2016}

\begin{tightemize}
\item \textbf{39K Downloads}
\item Designed both the UI and the icons used throughout the app
\item Illustrated marketing material used on the App Store.
\\
\item \textcolor{darkgray}{\textit{Github:} https://github.com/whoiswillma/Cello-Mute}
\item \textcolor{darkgray}{\textit{Technologies:} AudioKit, UIKit animations}
\end{tightemize}

\iffalse
%%%%%%%%%%%%%%%%%%%%%% COMMENTED OUT %%%%%%%%%%%%%%%%%%%%%%%%%
\runsubsection{Liloc}
\descript{Life by Location}
\location{Summer 2019 | Independent iOS App}
\begin{tightemize}
\item Extensive use of Core Data and related technologies
\item Use of custom view controller transitions and animations, both interactive and non-interactive.
\item Exploration of location-based APIs.
\\
\item \textcolor{darkgray}{\textit{Github: https://github.com/whoiswillma/Liloc}}
\item \textcolor{darkgray}{\textit{Technologies: \textbf{Core Data}, \textbf{Core Location}, NSFetchedResultsController, Swift 5 }}
\end{tightemize} 
\sectionsep
\fi

\iffalse
%%%%%%%%%%%%%%%%%%%%%% COMMENTED OUT %%%%%%%%%%%%%%%%%%%%%%%%%
\section{Course Scheduler}
\descript{Plan a Cornell Semester}
\descript{Spring 2019 | CS 3110 Project}
\begin{tightemize}
\item Arranges a schedule from chosen courses
\item Uses data from Cornell's API to schedule courses
\item Select between a distributed or condensed schedule
\\
\item \textcolor{darkgray}{\textit{Github: https://github.com/whoiswillma/cs3110-s2019-scheduler}}
\end{tightemize}
\sectionsep
\fi

\iffalse
%%%%%%%%%%%%%%%%%%%%%% COMMENTED OUT %%%%%%%%%%%%%%%%%%%%%%%%%
\section{Projects / Java}

\runsubsection{Critterworld}
\descript{Simulating Evolving Artificial Life}
\location{Fall 2018 | CS 2112 Project}
\begin{tightemize}
\item Parse arbitrarily complex AST from BNF specifications, apply transformations on the AST
\item Coordinate multiple clients over networking API
\item Designed a JavaFX GUI application with concurrency to view the state of the UI
\\
\item \textcolor{darkgray}{\textit{Technologies: Java FX, Java Concurrency and Networking, AST Parsing and Interpretation}}
\end{tightemize}
\sectionsep
\fi

\end{minipage} 
\end{document}